\documentclass[12pt, a4paper]{article}
\usepackage[utf8]{inputenc}
\usepackage[english]{babel}
\usepackage[T1]{fontenc}
\usepackage{lmodern}
\usepackage{amsmath}
%\usepackage{relsize}

%\usepackage{fixmath}
\usepackage{graphicx}
\usepackage{subcaption}
\usepackage[usenames,dvipsnames]{color}
%\usepackage[small,font=it]{caption}
\usepackage{amssymb}
%\usepackage{icomma}
%\usepackage{hyperref}
%\usepackage{mcode}
\usepackage{verbatim}
\usepackage{hyperref}
\usepackage{listings}
\usepackage{units}
\usepackage{fancyhdr}
\pagestyle{fancy}
\lhead{\footnotesize \parbox{11cm}{Classical transport from linearized Fokker-Planck collision operator}}
\rhead{\footnotesize \parbox{2cm}{Stefan Buller}}
\renewcommand\headheight{24pt}
%\usepackage{dirtytalk} \say{} command for quotations

\makeatletter
\newenvironment{tablehere}
  {\def\@captype{table}}
  {}

\newenvironment{figurehere}
  {\def\@captype{figure}}
  {}

\newsavebox\myboxA
\newsavebox\myboxB
\newlength\mylenA

\newcommand*\obar[2][0.75]{% OverBAR, adds bar over an element
    \sbox{\myboxA}{$\m@th#2$}%
    \setbox\myboxB\null% Phantom box
    \ht\myboxB=\ht\myboxA%
    \dp\myboxB=\dp\myboxA%
    \wd\myboxB=#1\wd\myboxA% Scale phantom
    \sbox\myboxB{$\m@th\overline{\copy\myboxB}$}%  Overlined phantom
    \setlength\mylenA{\the\wd\myboxA}%   calc width diff
    \addtolength\mylenA{-\the\wd\myboxB}%
    \ifdim\wd\myboxB<\wd\myboxA%
       \rlap{\hskip 0.5\mylenA\usebox\myboxB}{\usebox\myboxA}%
    \else
        \hskip -0.5\mylenA\rlap{\usebox\myboxA}{\hskip 0.5\mylenA\usebox\myboxB}%
    \fi}

\makeatother


%\def\equationautorefname{ekvation}
%\def\tableautorefname{tabell}
%\def\figureautorefname{figur}
%\def\sectionautorefname{sektion}
%\def\subsectionautorefname{sektion}

\DeclareMathOperator\erf{erf}
%\DeclareMathOperator\mod{mod}

\newcommand{\ordo}[1]{{\cal O}\left( #1 \right)}
\DeclareMathOperator{\sgn}{sgn}
%\renewcommand{\vec}[1]{\boldsymbol{#1}}
\newcommand{\im}{\ensuremath{\mathrm{i}}}
\newcommand{\e}{\ensuremath{\mathrm{e}}}
\newcommand{\E}{\ensuremath{\mathcal{E}}}
\newcommand{\p}{\ensuremath{\partial}}
\newcommand{\Z}{\ensuremath{\mathbb{Z}}}

\newcommand{\bra}[1]{\langle #1 \mid}
\newcommand{\ket}[1]{\mid #1 \rangle}
\newcommand\matris[4]{\ensuremath{\begin{pmatrix} #1 & #2 \\ #3 & #4\end{pmatrix}}}
\renewcommand{\d}{\ensuremath{\mathrm{d}}}
%\DeclareMathOperator*{\sgn}{sgn}
\newcommand{\todo}[1]{\textcolor{red}{#1}}
\newcommand{\lang}{\left\langle}
\newcommand{\rang}{\right\rangle}



%to get leftrightarrow over tensors of rank-2.
\def\shrinkage{2.1mu}
\def\vecsign{\mathchar"017E}
\def\dvecsign{\smash{\stackon[-1.95pt]{\mkern-\shrinkage\vecsign}{\rotatebox{180}{$\mkern-\shrinkage\vecsign$}}}}
\def\dvec#1{\def\useanchorwidth{T}\stackon[-4.2pt]{#1}{\,\dvecsign}}
\usepackage{stackengine}
\stackMath
\def\perfect{\textsc{Perfect}}


\lstset{language=[90]Fortran,
  basicstyle=\ttfamily,
  keywordstyle=\color{red},
  commentstyle=\color{green},
  morecomment=[l]{!\ }% Comment only with space after !
   frame=single,
   breaklines=true,
   postbreak=\raisebox{0ex}[0ex][0ex]{\ensuremath{\color{blue}\hookrightarrow\space}}
}

\usepackage[
    backend=biber,
    style=alphabetic,
    citestyle=authoryear,
    natbib=true,
    url=false,
    doi=false,
    isbn=false,
    eprint=false,
    sorting=none,
    maxcitenames=2,
    maxbibnames=3
]{biblatex} 

\AtEveryBibitem{\clearfield{month}}
\AtEveryCitekey{\clearfield{month}}
\AtEveryBibitem{\clearfield{issue}}
\AtEveryCitekey{\clearfield{issue}}

\bibliography{../../../plasma-bib} 

%=================================================================
\begin{document}
\section{Drift-Kinetic Classical transport}
For a mass-ratio expanded ion-impurity collision operator, we found that the classical transport of impurities in a stellarator can be comparable to the neoclassical transport. Thus, it would be of interest to have a general numerical tool to calculate the classical transport alongside the neoclassical transport from a drift-kinetic solver such as SFINCS.

In many ways, this is simpler than calculating the neoclassical transport, as the gyrophase dependent part of the distribution function is smaller than the gyrophase-independent in the expansion parameter $\rho/L$, and the gyrophase dependent part to required order is given entirely by the zeroth order gyrophase independent distribution. When $f_{a0}$ is a Maxwellian, the gyrophase dependent part is given by
\begin{equation}
\tilde{f}_{a1} = -\vec{\rho}_a(\gamma) \cdot \nabla f_{a0} = -\vec{\rho}(\gamma) \cdot \nabla \psi  \frac{\p f_{a0}}{\p \psi}, \label{eq:tildeF}
\end{equation}
where $\gamma$ is the gyrophase, $\psi$ a flux-label and $\vec{\rho}$ the gyroradius vector
\begin{equation}
\vec{\rho}_a = \rho_a (\vec{e}_2 \sin \gamma + \vec{e}_3 \cos\gamma),
\end{equation}
and we follow the same notation as in [Hazeltine \& Meiss, \emph{Plasma Confinement}, Dover Edition, chapter 4.2]. Note that $\tilde{f}_{a1}$ is highly anisotropic in velocity space, as $\rho_a = v_\perp/\Omega_a$ and $\gamma$ is a velocity space coordinate. 

The perpendicular friction force is calculated as
\begin{equation}
  \vec{R}_{a\perp} = \int \d^3 v m_a \vec{v}_\perp C[f_a]
\end{equation}
where
\begin{equation}
  \vec{v}_\perp = v_\perp \left(\vec{e}_2 \cos\gamma - \vec{e}_3 \sin \gamma\right),
\end{equation}
and $C[f_a] = \sum_b C[f_a,f_b]$ is the sum of of collisions with $a$ and all species, including self collisions $a=b$. We can rewrite the velocity space average above as
\begin{equation}
  \vec{R}_{a\perp} = \int \d v_\| v_\perp \d v_\perp\, \int \d\gamma m_a \vec{v}_\perp C[f_a].
\end{equation}
Splitting $C[f_a]$ into a gyrophase independent and dependent part $C[f_a]  = \bar{C[f_a]} + \tilde{C[f_a]}$, we see that only the latter contributes, and so
\begin{equation}
  \vec{R}_{a\perp} = \int \d v_\| v_\perp \d v_\perp\, \int \d\gamma m_a \vec{v}_\perp \tilde{C[f_a]}.
\end{equation}
We thus need to know $\tilde{C[f_a]}$.

\section{Linearized collision operator}
In drift-kinetics, $\tilde{f} \approx \tilde{f}_{a1} \sim \frac{\rho}{L} f_{a0} \ll f_{a0}$, and so we can linearize the collision operators as
\begin{equation}
C_l[f_a,f_b] = C[f_{a0},f_{b0}] +  C[f_{1a},f_{b0}] + C[f_{a0},f_{1b}],
\end{equation}
so that
\begin{equation}
C_l[f_a] = \sum_b \left(C[f_{a0},f_{b0}] +  C[f_{1a},f_{b0}] + C[f_{a0},f_{1b}]\right)
\end{equation}
is now a linear operator in terms of its actions on $f_{a1}$. The gyrophase dependent part of $C_l$ is now given by
\begin{equation}
\tilde{C_l[f_a]} = \sum_b \left(C[\tilde{f}_{1a},f_{b0}] + C[f_{a0},\tilde{f}_{1b}]\right),
\end{equation}
where we need to evaluate collisions with a Maxwellian background and inversely. 

\section{Fokker-Planck operator}
For the Fokker-Planck operator, we have [eq. 3.40 in Helander \& Sigmar, \emph{Collisional transport magnetized plasmas}] that
\begin{equation}
C[f_{a},f_{b0}] = \nu_D^{ab} + \frac{1}{v^2} \frac{\p}{\p v} \left[v^3 \left( \frac{m_a}{m_a + m_b} \nu_s^{ab} f_a + \frac{1}{2} \nu_\|^{ab} v \frac{\p f_a}{\p v}\right)\right]
\end{equation}
where the collision frequencies are known. The resulting contribution to the friction force is
\begin{equation}
  \int m_a \vec{v} C[f_{a},f_{b0}] = - \int m_a \vec{v} \nu_s^{ab} f_{a} \d^3 v.
\end{equation}

The field-particle term would be more complicated, but its contribution to the friction force can be obtained from momentum conservation
\begin{equation}
\int m_a \vec{v} C[f_{a0},f_{b}] = -\int m_b \vec{v} C[f_{b},f_{a0}] = \int m_b \vec{v} \nu_s^{ba} f_{b1} \d^3 v.
\end{equation}
The total perpendicular friction force from the linearized operator is thus
\begin{equation}
  \vec{R}_{a\perp} = -\sum_b \left( \int  \d^3 v \, \nu_s^{ab} m_a \vec{v}_\perp \tilde{f}_{a1} - \int  \d^3 v\, \nu_s^{ba}m_b  \vec{v}_\perp  \tilde{f}_{b1} \right),
\end{equation}
where
\begin{align}
  \nu_s^{ab}(v) = \hat{\nu}_{ab} \frac{2 T_a}{T_b} \left(1 + \frac{m_b}{m_a} \right) \frac{G(x_b)}{x_a}
\end{align}
only depends on $v$. Here $G(x)$ is the Chandrasekhar function,
\begin{equation}
G(x) = \frac{\erf{(x)}  - x \erf'{(x)}}{2x^2} = \frac{\erf{(x)}  - \frac{2}{\sqrt{\pi}}x \e^{-x^2}}{2x^2},
\end{equation}
and $x_a = v/v_{Ta}$ with $v_{Ta} = \sqrt{2T_a/m_a}$. The collision frequency $\hat{\nu}_{ab}$ is
\begin{equation}
\hat{\nu}_{ab} = \frac{n_b Z_a Z_b e^4 \ln \Lambda}{4\pi \epsilon_0^2 m_a^2 v_{Ta}^3}.
\end{equation}

Using $\{v,v_\|, \gamma\}$ as our velocity space coordinates, we have
\begin{align}
m_a \int_0^\infty v  \d v \, \nu_s^{ab} \int_{-v}^v \d v_\| \, \oint \d\gamma\,  \vec{v}_\perp \tilde{f}_{a1},
\end{align}
and inserting our expression for $\tilde{f}_{a1}$, we get
\begin{align}
-\frac{m_a}{\Omega_a} \int_0^\infty \d v \, v \nu_s^{ab} \frac{\p f_{a0}}{\p \psi} \int_{-v}^v \d v_\|\, (v^2 - v_\|^2) \oint \d\gamma\,   \hat{v}_\perp \hat{\rho} \cdot \nabla \psi.
\end{align}

The $\gamma$ integral does not depend on any other velocity space coordinate, and gives
\begin{equation}
\oint \d\gamma \, \hat{v}_\perp \hat{\rho} \cdot \nabla \psi = - \pi\vec{b} \times \nabla \psi.
\end{equation}
The $v_\|$ integral gives
\begin{align}
 \int_{-v}^v \d v_\|\, (v^2 - v_\|^2) = \frac{4v^3}{3},
\end{align}
and thus
\begin{align}
m_a \int_0^\infty v  \d v \, \nu_s^{ab} \int_{-v}^v \d v_\| \, \oint \d\gamma\,  \vec{v}_\perp \tilde{f}_{a1} = \vec{b} \times \nabla \psi \frac{4\pi m_a}{3\Omega_a} \int_0^\infty \d v \, v^4 \nu_s^{ab} \frac{\p f_{a0}}{\p \psi}.
\end{align}
The $\psi$-derivative is
\begin{equation}
\frac{\p f_{a0}}{\p \psi} = f_{a0} \left[A_{1a} - \frac{5}{2} A_{2a} + \frac{m_av^2}{2T_a}  A_{2a}\right], \label{eq:dpsifM}
\end{equation}
and so
\begin{equation}
\begin{aligned}
  &m_a \int_0^\infty v  \d v \, \nu_s^{ab} \int_{-v}^v \d v_\| \, \oint \d\gamma\,  \vec{v}_\perp \tilde{f}_{a1} \\
  =& \vec{b} \times \nabla \psi \frac{4\pi m_a}{3\Omega_a} \left[\left(A_{1a} - \frac{5}{2} A_{2a}\right) \int_0^\infty \d v \, v^4 \nu_s^{ab}  f_{a0} + \frac{m_a}{2T_a} A_{2a} \int_0^\infty \d v \, v^6 \nu_s^{ab}  f_{a0}  \right].
\end{aligned}
\end{equation}

Introducing
\begin{align}
  D^1_{ab} = \frac{4\pi m_a^2}{3 Z_a e} \int_0^\infty \d v \, v^4 \nu_s^{ab}  f_{a0} \\
  D^2_{ab} = \frac{4\pi m_a^2}{3 Z_a e} \frac{m_a}{2T_a} \int_0^\infty \d v \, v^6 \nu_s^{ab}  f_{a0},
\end{align}
the perpendicular friction becomes
\begin{equation}
 \vec{R}_{a\perp} = \frac{\vec{B} \times \nabla \psi}{B^2} \sum_b \left[D^1_{ba}\left(A_{1b} - \frac{5}{2} A_{2b}\right) + D^2_{ba}A_{2b}  -D^1_{ab}\left(A_{1a} - \frac{5}{2} A_{2a}\right) - D^2_{ab}A_{2a}\right].
\end{equation}
The resulting classical transport is thus
\begin{equation}
  \begin{aligned}
    \Gamma_a^{\text{C}} = &\frac{1}{Z_a e} \lang B^{-2} \left(\vec{B} \times \nabla \psi\right) \cdot \vec{R}_a \rang \\
    = & 
  \frac{1}{Z_a e}\lang \frac{|\nabla \psi|^2}{B^2} \sum_b \left[D^1_{ba}\left(A_{1b} - \frac{5}{2} A_{2b}\right) + D^2_{ba}A_{2b}  -D^1_{ab}\left(A_{1a} - \frac{5}{2} A_{2a}\right) - D^2_{ab}A_{2a}\right] \rang.
\end{aligned}
\end{equation}

All that now remains is to evaluate the $D^1_{ab}$ and $D^2_{ab}$ coefficients. Using
\begin{align}
  f_{a0} &= n_a \frac{1}{\pi^{3/2} v_{Ta}^3} \e^{-v^2/v_{Ta}^2} = n_a \frac{1}{\pi^{3/2} v_{Ta}^3} \e^{-x_b^2 \frac{T_b m_a}{T_a m_b}} \\
  \nu_{s}^{ab} &= \frac{n_b Z_b^2 Z_a^2 \ln \Lambda}{4 \pi \epsilon_0^2 m_a^2 v_{Ta}^3} \frac{2T_a}{T_b} \left(1 + \frac{m_b}{m_a}\right) \frac{G(x_b)}{x_a} \\
  x_b &= \frac{v}{v_{Tb}} \\
  v_{Ta} &= \sqrt{2T_a/m_a},
\end{align}
we find that
\begin{align}
  D^1_{ab} = \frac{\sqrt{2}}{3\pi^{3/2}} \frac{n_a n_b  e^3 \ln \Lambda}{\epsilon_0^2 } Z_b^2 Z_a \frac{T_b}{T_a^{3/2}} \frac{m_a^{3/2}}{m_b}\left(1 + \frac{m_a}{m_b} \right) \int \d x_b\, x_b^3 G(x_b) \e^{-x_b^2 \frac{T_b m_a}{T_a m_b}} \\
  D^2_{ab} = \frac{\sqrt{2}}{3\pi^{3/2}} \frac{n_a n_b e^3 \ln \Lambda}{\epsilon_0^2 } Z_b^2 Z_a  \frac{T_b^2}{T_a^{5/2}} \frac{m_a^{5/2}}{m_b^2} \left(1 + \frac{m_a}{m_b}\right) \int \d x_b\, x_b^5 G(x_b) \e^{-x_b^2 \frac{T_b m_a}{T_a m_b}} \\
  D^1_{ba} = \frac{\sqrt{2}}{3\pi^{3/2}} \frac{n_a n_b  e^3 \ln \Lambda}{\epsilon_0^2 } Z_a^2 Z_b \frac{T_a}{T_b^{3/2}} \frac{m_b^{3/2}}{m_a}\left(1 + \frac{m_b}{m_a} \right) \int \d x_a\, x_a^3 G(x_a) \e^{-x_a^2 \frac{T_a m_b}{T_b m_a}} \\
D^2_{ba} = \frac{\sqrt{2}}{3\pi^{3/2}} \frac{n_a n_b e^3 \ln \Lambda}{\epsilon_0^2 }  Z_a^2 Z_b \frac{T_a^2}{T_b^{5/2}} \frac{m_b^{5/2}}{m_a^2} \left(1 + \frac{m_b}{m_a}\right) \int \d x_a\, x_a^5 G(x_a) \e^{-x_a^2 \frac{T_a m_b}{T_b m_a}} 
\end{align}
where the integrals have to be evaluated numerically for each $\frac{T_b m_a}{T_a m_b}$. 

Defining
\begin{align}
  F(y) &= y\int_0^{\infty} \d x\, x^3 G(x) \e^{-yx^2}  = \frac{1}{4(1+y)^{3/2}} \\
  H(y) &= y^2\int_0^{\infty} \d x\, x^5 G(x) \e^{-yx^2} =  \frac{5y + 2}{8(1+y)^{5/2}}\\
  F_2(y) &= H(y)-F(y) =  \frac{3y}{8(1+y)^{5/2}} = \frac{3y}{2(1+y)} F(y)\\
\end{align}
we have
\begin{align}
  D^1_{ab} = \frac{\sqrt{2}}{3\pi^{3/2}} \frac{n_a n_b  e^3 \ln \Lambda}{\epsilon_0^2 }  Z_a Z_b^2\frac{m_a^{1/2}}{T_a^{1/2}}\left(1 + \frac{m_a}{m_b} \right) F\left(\frac{T_b m_a}{T_a m_b}\right) \\
  D^2_{ab} = \frac{\sqrt{2}}{3\pi^{3/2}} \frac{n_a n_b e^3 \ln \Lambda}{\epsilon_0^2 }  Z_a Z_b^2 \frac{m_a^{1/2}}{T_a^{1/2}} \left(1 + \frac{m_a}{m_b}\right) H\left(\frac{T_b m_a}{T_a m_b}\right) \\
  D^1_{ba} = \frac{\sqrt{2}}{3\pi^{3/2}} \frac{n_a n_b  e^3 \ln \Lambda}{\epsilon_0^2 } Z_a^2 Z_b \frac{m_b^{1/2}}{T_b^{1/2}} \left(1 + \frac{m_b}{m_a} \right)  F\left(\frac{T_a m_b}{T_b m_a}\right) \\
D^2_{ba} = \frac{\sqrt{2}}{3\pi^{3/2}} \frac{n_a n_b e^3 \ln \Lambda}{\epsilon_0^2 }  Z_a^2 Z_b \frac{m_b^{1/2}}{T_b^{1/2}}  \left(1 + \frac{m_b}{m_a}\right) H\left(\frac{T_a m_b}{T_b m_a}\right)
\end{align}

Thus, the classical flux is given by
\begin{equation}
  \begin{aligned}
   \Gamma_a^{\text{C}} = \frac{\sqrt{2}}{3\pi^{3/2}} \frac{ e^2 \ln \Lambda}{\epsilon_0^2 } 
    \lang \frac{|\nabla \psi|^2}{B^2} n_a \sum_b Z_b n_b \right.\left[\phantom{\frac{m_b^{3/2}}{m_a}} \right.
      &+ Z_a  \frac{m_b^{1/2}}{T_b^{1/2}} \left(1 + \frac{m_b}{m_a} \right)  F\left(\frac{T_a m_b}{T_b m_a}\right) \left(A_{1b} - \frac{5}{2} A_{2b}\right) \\
      &+ Z_a  \frac{m_b^{1/2}}{T_b^{1/2}} \left(1 + \frac{m_b}{m_a}\right) H\left(\frac{T_a m_b}{T_b m_a}\right) A_{2b} \\
      &-Z_b\frac{m_a^{1/2}}{T_a^{1/2}}\left(1 + \frac{m_a}{m_b} \right) F\left(\frac{T_b m_a}{T_a m_b}\right) \left(A_{1a} - \frac{5}{2} A_{2a}\right) \\ &\left.\left.
      - Z_b \frac{m_a^{1/2}}{T_a^{1/2}}  \left(1 + \frac{m_a}{m_b}\right) H\left(\frac{T_b m_a}{T_a m_b}\right)A_{2a} \right] \rang.
\end{aligned}
\end{equation}

To see that the radial electric field does not contribute for any mass-ratio, we note that
\begin{equation}
F(y^{-1}) = y^{3/2} F(y).
\end{equation}
This allows us to combine $D^1_{ab}$ and $D^1_{ba}$ in a way that makes it explicit that the radial electric field does not contribute. We can treat the $H(y)$ terms in a similar way by writing $H(y) = F(y) + F_2(y)$ and noting that
\begin{equation}
F_2(y^{-1}) = y^{1/2} F_2(y).
\end{equation}
Thus
\begin{equation}
  \begin{aligned}
   \Gamma_a^{\text{C}} = \frac{\sqrt{2}}{3\pi^{3/2}} \frac{ e^2 \ln \Lambda}{\epsilon_0^2 } 
    \lang \frac{|\nabla \psi|^2}{B^2} n_a \sum_b Z_b n_b \right.\left[\phantom{\frac{m_b^{3/2}}{m_a}} \right.
      &+ Z_a  \frac{m_b^{1/2}}{T_b^{1/2}} \left(1 + \frac{m_b}{m_a} \right)  F\left(\frac{T_a m_b}{T_b m_a}\right) \left(A_{1b} - \frac{3}{2} A_{2b}\right) \\
      &+ Z_a  \frac{m_b^{1/2}}{T_b^{1/2}} \left(1 + \frac{m_b}{m_a}\right) F_2\left(\frac{T_a m_b}{T_b m_a}\right) A_{2b} \\
      &-Z_b\frac{m_b^{1/2}}{T_b^{1/2}}\left(1 + \frac{m_b}{m_a} \right) F\left(\frac{T_a m_b}{T_b m_a}\right) \frac{T_a}{T_b}\left(A_{1a} - \frac{3}{2} A_{2a}\right) \\ &\left.\left.
      - Z_b \frac{m_b^{1/2}}{T_b^{1/2}}  \left(1 + \frac{m_b}{m_a}\right) F_2\left(\frac{T_a m_b}{T_b m_a}\right) \frac{m_a}{m_b} A_{2a} \right] \rang,
\end{aligned}
\end{equation}
or
\begin{equation}
  \begin{aligned}
   \Gamma_a^{\text{C}} = \frac{\sqrt{2}}{3\pi^{3/2}} \frac{ e^2 \ln \Lambda}{\epsilon_0^2 } 
    \lang \frac{|\nabla \psi|^2}{B^2} n_a \sum_b Z_b n_b \frac{m_b^{1/2}}{T_b^{1/2}} \left(1 + \frac{m_b}{m_a} \right) F\left(\frac{T_a m_b}{T_b m_a}\right)\right.\left[\vphantom{\frac{T_am_b}{T_bm_a}} \right.
      + &Z_a     \left(A_{1b} - \frac{3}{2} A_{2b}\right) \\
      + &Z_a  \frac{3}{2} \frac{T_a m_b}{T_b m_a + T_a m_b} A_{2b} \\
      - &Z_b  \frac{T_a}{T_b}\left(A_{1a} - \frac{3}{2} A_{2a}\right) \\ 
      - &\left.\left.Z_b \frac{3}{2} \frac{T_a m_a}{T_b m_a + T_a m_b}  A_{2a} \right] \rang,
\end{aligned} \label{eq:fluxg}
\end{equation}
where we have factored out all the common factors and used $F_2(y) = 3y F(y)/(2+2y)$.


\clearpage
\subsection{Heavy, $Z\gg 1$ impurity}
We check our general expression \eqref{eq:fluxg} against the flux of a heavy, high-$Z$ impurity. Taking $T_a \sim T_b$ and assuming $m_a \gg m_b$, we have that
\begin{equation}
F\left(\frac{T_a m_b}{T_b m_a}\right) = \frac{1}{4},
\end{equation}
and thus
\begin{equation}
  \begin{aligned}
   \Gamma_a^{\text{C}} = \frac{\sqrt{2}}{12\pi^{3/2}} \frac{ e^2 \ln \Lambda}{\epsilon_0^2 } 
    \lang \frac{|\nabla \psi|^2}{B^2} n_a \sum_b Z_b n_b \frac{m_b^{1/2}}{T_b^{1/2}} \right.\left[\vphantom{\frac{T_am_b}{T_bm_a}} \right.
      &\left.\left.Z_a     \left(A_{1b} - \frac{3}{2} A_{2b}\right)  - Z_b  \frac{T_a}{T_b}A_{1a} \right] \rang.
\end{aligned} \label{eq:fluxheavy}
\end{equation}
Using $Z_a \gg 1$ to drop terms in $A_{1a}$ that are $Z_a^0$, we thus get
\begin{equation}
  \begin{aligned}
     \Gamma_a^{\text{C}} = \frac{1}{6\pi^{3/2}} \frac{  Z_a e^2 \ln \Lambda}{\epsilon_0^2 } 
    \lang n_a \frac{|\nabla \psi|^2}{B^2} \sum_b \frac{Z_b n_b }{v_{Tb}} \left[ A_{1b} - \frac{3}{2} A_{2b} - \frac{Z_b e}{T_b}\frac{\d \Phi}{\d \psi}\right] \rang,
\end{aligned}
\end{equation}
which for a single ion $b = i$ with flux-function density gives
\begin{equation}
  \begin{aligned}
\Gamma_a^{\text{C}} = \frac{1}{6\pi^{3/2}} \frac{ Z_a Z_i n_i e^2 \ln \Lambda}{\epsilon_0^2 v_{Ti}} 
    \lang n_a  \frac{|\nabla \psi|^2}{B^2} \rang \left[ A_{1i} - \frac{3}{2} A_{2i} - \frac{Z_i e}{T_b} \frac{\d \Phi}{\d \psi}\right].
  \end{aligned}
\end{equation}
Introducing $\tau_{ia}^{-1} = \frac{Z_a^2 Z_i^2 n_a e^4 \ln \Lambda}{3\pi^{3/2} m_i^2 \epsilon_0^2 v_{Ti}^3}$, we find
\begin{equation}
  \begin{aligned}
\Gamma_a^{\text{C}} =  \frac{m_i n_i}{Z_a e n_a\tau_{ia}}
    \lang n_a  \frac{|\nabla \psi|^2}{B^2} \rang \frac{T_i}{Z_i e}\left[ A_{1i} - \frac{3}{2} A_{2i} - \frac{Z_i e}{T_i} \frac{\d \Phi}{\d \psi}\right],
  \end{aligned}
\end{equation}
which agrees with the expression we obtained from a mass-ratio expanded collision operator in the $Z\gg 1$ limit.

\subsection{Gradients instead of thermodynamic forces}
The thermodynamic forces are defined so that the gradients of the Maxwellian satisfies \eqref{eq:dpsifM}. As a result, different thermodynamic forces are obtained depending on both what we include in the Maxwellian, and what coordinates we keep fixed when performing the partial derivatives.

For compatibility with SFINCS notation, we define $f_0$ containing potential variation on the flux-surface, and $f_M$ without.
\begin{align}
  f_{aM} &= N_a(\psi) \left(\frac{m_a}{2\pi T_a} \right)^{3/2} \exp{\left(-\frac{m_a v^2}{2T_a} -\frac{Z_a e\lang \Phi \rang}{T_a}\right)} \\
  f_{a0} &= N_a(\psi) \left(\frac{m_a}{2\pi T_a} \right)^{3/2} \exp{\left(-\frac{m_a v^2}{2T_a} - \frac{Z_a e(\lang \Phi \rang + \tilde{\Phi})}{T_a}\right)} = f_{aM} \e^{- \frac{Z_a e\tilde{\Phi}}{T_a}}.
\end{align}
When the appropriate flags are set in SFINCS, $f_{a0}$ will be used instead of $f_{aM}$. As the $f_{aM}$ case can be recovered from the $f_{a0}$ case by setting $\tilde{\Phi} = 0$, we will calculate the thermodynamic forces for $f_{a0}$ only. 

The gradients in \eqref{eq:tildeF} are performed with the total energy fixed, including $Z_a e\tilde{\Phi}$. Thus
\begin{equation}
 \frac{\p_\psi f_{a0}}{f_{a0}} = \frac{\d_\psi N_a}{N_a} - \frac{3}{2}\frac{\d_\psi T_a}{T_a} + \frac{1}{T_a}\left(\frac{m_a v^2}{2} + Z_a e (\lang \Phi \rang + \tilde{\Phi})\right) \frac{\d_\psi T_a}{T_a}, \label{eq:dpsifM2}
\end{equation}
where
\begin{align}
  N_a = n_a \exp{\left(\frac{Z_a e(\lang \Phi \rang + \tilde{\Phi})}{T_a} \right)}
\end{align}
and
\begin{align}
  \frac{\d_\psi N_a}{N_a} = \frac{\p_\psi n_a}{n_a} + \frac{Z_a e \p_\psi (\lang \Phi \rang + \tilde{\Phi})}{T_a} - \frac{Z_a e (\lang \Phi \rang + \tilde{\Phi})}{T_a} \frac{\d_\psi T_a}{T_a},
\end{align}
so that
\begin{equation}
\frac{\p_\psi f_{a0}}{f_{a0}} = \frac{\p_\psi n_a}{n_a} + \frac{Z_a e \p_\psi (\lang \Phi \rang + \tilde{\Phi})}{T_a} - \frac{3}{2}\frac{\d_\psi T_a}{T_a} + \frac{m_a v^2}{2T_a}  \frac{\d_\psi T_a}{T_a},
\end{equation}
from which we identify
\begin{align}
  A_{1a} &= \frac{\p_\psi p_a}{p_a} + \frac{Z_a e}{T_a} \p_\psi (\lang \Phi \rang + \tilde{\Phi}) \\
  A_{2a} &= \frac{\d_\psi T_a}{T_a},
\end{align}
as expected. From this and the form of the classical flux in \eqref{eq:fluxg}, we can see that the radial electric field does not contribute to the classical flux, since $A_{1}$ only enters as
\begin{equation}
Z_a A_{1b} - Z_b \frac{T_a}{T_b} A_{1a}.
\end{equation}
Furthermore, as expected, when expressed in terms of physical densities, the thermodynamic forces do not depend on the value of $\Phi$, so is gauge-invariant, as they should be. 

However, in SFINCS, we specify $\d_\psi n_a \e^{\left(\frac{Z_a e \tilde{\Phi}}{T_a} \right)}$ rather than the density gradient $\p_\psi n_a$. Defining
\begin{equation}
h_a =   n_a \e^{\left(\frac{Z_a e \tilde{\Phi}}{T_a} \right)},
\end{equation}
we get $A_{1a}$ in terms of SFINCS inputs as
\begin{align}
  A_{1a} = \frac{\d_\psi h_a}{h_a} + \frac{Z_a e}{T_a} \p_\psi \tilde{\Phi} + \frac{\p_\psi T_a}{T_a} + \frac{Z_a e \tilde{\Phi}}{T_a} \frac{\d_\psi T_a}{T_a},
\end{align}
with the resulting classical particle flux
\begin{equation}
  \begin{aligned}
    &\Gamma_a^{\text{C}} \frac{3\pi^{3/2}} {\sqrt{2}}\frac{\epsilon_0^2 }{ e^2 \ln \Lambda} \\
    =&  
    \sum_b Z_b \frac{m_b^{1/2}}{T_b^{1/2}} \left(1 + \frac{m_b}{m_a} \right) F\left(\frac{T_a m_b}{T_b m_a}\right) \left( \vphantom{ \lang \frac{|\nabla \psi|^2}{B^2} n_a n_b \rang}\right.\\
      &  \lang \frac{|\nabla \psi|^2}{B^2} n_a n_b \rang \left[Z_a     \left(\frac{\d_\psi h_b}{h_b}  - \frac{1}{2} \frac{\p_\psi T_b}{T_b}\right) 
      + Z_a  \frac{3}{2} \frac{T_a m_b}{T_b m_a + T_a m_b} \frac{\d_\psi T_b}{T_b} \right.\\
      & - Z_b  \frac{T_a}{T_b}\left(\frac{\d_\psi h_a}{h_a}  - \frac{1}{2} \frac{\p_\psi T_a}{T_a}\right) 
      - \left.Z_b \frac{3}{2} \frac{T_a m_a}{T_b m_a + T_a m_b}  \frac{\d_\psi T_a}{T_a} \right] \\
      & +\left. \lang \frac{|\nabla \psi|^2}{B^2} n_a n_b \tilde{\Phi}\rang Z_a Z_b \frac{e}{T_b} \left( \frac{\d_\psi T_b}{T_b} - \frac{\d_\psi T_a}{T_a} \right)\right),
\end{aligned} \label{eq:fluxgS}
\end{equation}
where the radial electric field has been cancelled, as anticipated.

For $\tilde{\Phi} = 0$, we get that $h_a = n_a$, the last term disappears, and $n_a$ and $n_b$ will be flux-functions, resulting in

\begin{equation}
  \begin{aligned}
    &\Gamma_a^{\text{C}} \frac{3\pi^{3/2}} {\sqrt{2}}\frac{\epsilon_0^2 }{ e^2 \ln \Lambda} \\
    =&  
    \sum_b Z_b  n_a n_b \frac{m_b^{1/2}}{T_b^{1/2}} \left(1 + \frac{m_b}{m_a} \right) F\left(\frac{T_a m_b}{T_b m_a}\right) \lang \frac{|\nabla \psi|^2}{B^2} \rang\\
      &   \left[Z_a     \left(\frac{\d_\psi n_b}{n_b}  - \frac{1}{2} \frac{\p_\psi T_b}{T_b}\right) \right.
      + Z_a  \frac{3}{2} \frac{T_a m_b}{T_b m_a + T_a m_b} \frac{\d_\psi T_b}{T_b} \\
      & - Z_b  \frac{T_a}{T_b}\left(\frac{\d_\psi n_a}{n_a}  - \frac{1}{2} \frac{\p_\psi T_a}{T_a}\right) 
      - \left.Z_b \frac{3}{2} \frac{T_a m_a}{T_b m_a + T_a m_b}  \frac{\d_\psi T_a}{T_a} \right] 
\end{aligned} \label{eq:fluxgSff}
\end{equation}

% \section{Kinetic definition}
% The transport in SFICNS is not defined in terms of parallel or perpendicular friction forces, but in terms of moments of a distribution function. It would thus be nice to have a kinetic definition of classical fluxes.

% One can attempt to define
% \begin{equation}
% \Gamma^C_a = \lang \int \d^3 v \tilde{f}_a \vec{v} \cdot \nabla \psi \rang.
% \end{equation}

% However, using
% Using $\tilde{f}_a = -\vec{\rho}_a \cdot \nabla \psi \frac{\p f_{a0}}{\p \psi}$, we have that
% \begin{equation}
% \Gamma^C_a = -\frac{1}{\Omega_a}\lang \int_0^\infty v  \d v \, \frac{\p f_{a0}}{\p \psi} \int_{-v}^v \d v_\| (v^2 - v_\|^2)\, \left [ \oint \d\gamma\   \hat{\rho}_a  \hat{v}_\perp \right] : \nabla \psi \nabla \psi \rang,
% \end{equation}
% which can be seen to be exactly zero, 
% \begin{equation}
% \Gamma^C_a = -\frac{1}{\Omega_a}\lang \int_0^\infty v  \d v \, \frac{\p f_{a0}}{\p \psi} \int_{-v}^v \d v_\| (v^2 - v_\|^2)\, \pi [\vec{e}_2 \vec{e}_3 - \vec{e}_3 \vec{e}_2]  : \nabla \psi \nabla \psi \rang = 0
% \end{equation}
% since the "magnetization flow" is in the diamagnetic direction. Thus: no kinetic version yet.

\section{Implementation in SFINCS}
\todo{Note: flux-label normalization in eq.\ 165 in the technical SFINCS documentation lacks a factor $\bar{R}$?}

Returning to the \eqref{eq:fluxgS}, we need to translate it to the dimensionless quantities used in SFINCS. The dimensionless quantities in SFINCS are
\begin{align}
  \hat{T}_a \equiv \frac{T_a}{\bar{T}} \\
  \hat{m}_a \equiv \frac{m_a}{\bar{m}} \\
  \hat{n}_a \equiv \frac{n_a}{\bar{n}} \\
  \hat{B} \equiv \frac{B}{\bar{B}} \\
  \hat{\psi} \equiv \frac{\psi}{\bar{R}^2\bar{B}} \\
  \hat{\Phi} \equiv \frac{\Phi}{\bar{\Phi}} \\
  \bar{v} \equiv \sqrt{\frac{2\bar{T}}{\bar{m}}} \\
  \alpha \equiv \frac{e\bar{\Phi}}{\bar{T}},
\end{align}
and we wish to introduce a normalized classical flux as
\begin{equation}
\hat{\Gamma}_a^{\text{C}} = \frac{\Gamma_a^{\text{C}} }{\bar{n} \bar{v}\bar{R}\bar{B}} = \frac{\bar{R}}{\bar{n}\bar{v}} \frac{1}{Z_a e} \lang B^{-2} \left(\vec{B} \times \nabla \hat{\psi}\right) \cdot \vec{R}_a \rang.
\end{equation}
With this, we obtain
\begin{equation}
  \begin{aligned}
    &\hat{\Gamma}_a^{\text{C}} 
    =  \frac{\sqrt{2}}{3\pi^{3/2}}\frac{ e^2 \ln \Lambda}{\epsilon_0^2 } \frac{1}{\bar{v} \bar{R} \bar{B}^2} \frac{\sqrt{\bar{m}}}{\sqrt{\bar{T}}} \bar{n}
    \sum_b Z_b \frac{\hat{m}_b^{1/2}}{\hat{T}_b^{1/2}} \left(1 + \frac{m_b}{m_a} \right) F\left(\frac{T_a m_b}{T_b m_a}\right) \left( \vphantom{ \lang \frac{\bar{R}^2|\nabla \hat{\psi}|^2}{\hat{B}^2} \hat{n}_a \hat{n}_b \rang}\right.\\
      &  \lang \frac{\bar{R}^2|\nabla \hat{\psi}|^2}{\hat{B}^2} \hat{n}_a \hat{n}_b \rang \left[Z_a     \left(\frac{\d_{\hat{\psi}} h_b}{h_b}  - \frac{1}{2} \frac{\p_{\hat{\psi}} T_b}{T_b}\right) 
      + Z_a  \frac{3}{2} \frac{T_a m_b}{T_b m_a + T_a m_b} \frac{\d_{\hat{\psi}} T_b}{T_b}\right. \\
      & - Z_b  \frac{T_a}{T_b}\left(\frac{\d_{\hat{\psi}} h_a}{h_a}  - \frac{1}{2} \frac{\p_{\hat{\psi}} T_a}{T_a}\right) 
      - \left.Z_b \frac{3}{2} \frac{T_a m_a}{T_b m_a + T_a m_b}  \frac{\d_{\hat{\psi}} T_a}{T_a} \right] \\
      & +\left. \lang \frac{\bar{R}^2|\nabla \hat{\psi}|^2}{\hat{B}^2} \hat{n}_a \hat{n}_b \hat{\tilde{\Phi}}\rang Z_a Z_b \frac{\alpha}{\hat{T_b}} \left[ \frac{\d_{\hat{\psi}} T_b}{T_b} - \frac{\d_{\hat{\psi}} T_a}{T_a} \right]\right),
\end{aligned} 
\end{equation}
where the prefactor of all the dimensional quantities is simplified by introducing
\begin{align}
  \bar{\nu} &\equiv \frac{\sqrt{2}}{12 \pi^{3/2}} \frac{\bar{n} e^4 \ln \Lambda}{\epsilon_0^2 \bar{m}^{1/2} \bar{T}^{3/2}}  \\
  \nu_n &\equiv \bar{\nu} \frac{\bar{R}}{\bar{v}} \implies \frac{e^2 \ln \Lambda}{\epsilon_0^2} = \frac{12 \pi^{3/2} \sqrt{\bar{m}} \bar{T}^{3/2}}{\sqrt{2} e^2 \bar{n}}\frac{\bar{v}}{\bar{R}} \nu_n
\end{align}
so that
\begin{equation}
  \frac{\sqrt{2}}{3\pi^{3/2}}\frac{ e^2 \ln \Lambda}{\epsilon_0^2 } \frac{1}{\bar{v} \bar{R} \bar{B}^2} \frac{\sqrt{\bar{m}}}{\sqrt{\bar{T}}} \bar{n}  =
  \frac{4\bar{m} \bar{T}}{e^2 \bar{B}^2}\frac{1}{\bar{R}^2} \nu_n.
\end{equation}
With $\Delta \equiv \frac{\sqrt{2\bar{m}\bar{T}}}{e\bar{R}\bar{B}}$, the prefactor finally becomes
  \begin{equation}
\frac{4\bar{m} \bar{T}}{e^2 \bar{B}^2}\frac{1}{\bar{R}^2} \nu_n = 2 \Delta^2 \nu_n,
\end{equation}
and the classical particle flux in SFINCS is thus
\begin{equation}
  \begin{aligned}
    &\hat{\Gamma}_a^{\text{C}}  
    =  2 \Delta \nu_n
    \sum_b Z_b \frac{\hat{m}_b^{1/2}}{\hat{T}_b^{1/2}} \left(1 + \frac{m_b}{m_a} \right) F\left(\frac{T_a m_b}{T_b m_a}\right) \left( \vphantom{ \lang \frac{\bar{R}^2|\nabla \hat{\psi}|^2}{\hat{B}^2} \hat{n}_a \hat{n}_b \rang}\right.\\
      &  \lang \frac{\bar{R}^2|\nabla \hat{\psi}|^2}{\hat{B}^2} \hat{n}_a \hat{n}_b \rang \left[Z_a     \left(\frac{\d_{\hat{\psi}} h_b}{h_b}  - \frac{1}{2} \frac{\p_{\hat{\psi}} T_b}{T_b}\right) 
      + Z_a  \frac{3}{2} \frac{T_a m_b}{T_b m_a + T_a m_b} \frac{\d_{\hat{\psi}} T_b}{T_b}\right. \\
      & - Z_b  \frac{T_a}{T_b}\left(\frac{\d_{\hat{\psi}} h_a}{h_a}  - \frac{1}{2} \frac{\p_{\hat{\psi}} T_a}{T_a}\right) 
      - \left.Z_b \frac{3}{2} \frac{T_a m_a}{T_b m_a + T_a m_b}  \frac{\d_{\hat{\psi}} T_a}{T_a} \right] \\
      & +\left. \lang \frac{\bar{R}^2|\nabla \hat{\psi}|^2}{\hat{B}^2} \hat{n}_a \hat{n}_b \hat{\tilde{\Phi}}\rang Z_a Z_b \frac{\alpha}{\hat{T_b}} \left[ \frac{\d_{\hat{\psi}} T_b}{T_b} - \frac{\d_{\hat{\psi}} T_a}{T_a} \right]\right).
\end{aligned} \label{eq:fluxgS2}
\end{equation}

Note that the only part of the above calculation that depends on SFINCS output is the perturbed potential $\tilde{\Phi}$. If $\tilde{\Phi}$ is not included in the SFINCS calculation, we are thus be able to calculate the classical flux directly from the inputs.
\end{document}
